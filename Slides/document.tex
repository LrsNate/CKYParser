\documentclass{beamer}

\usepackage[francais]{babel}
\usepackage[utf8]{inputenc}
\usepackage[T1]{fontenc}
\usepackage{enumitem}
\usepackage{xcolor}
\usepackage{qtree}

\begin{document}
\begin{frame}
\frametitle{Introduction : but de l'analyse syntaxique}
\pause
\textbf{Objectif} : mettre en \'evidence la structure d'un \'enonc\'e.

\vspace{1cm}
\pause

\textbf{Buts} :
\pause

\begin{itemize}
\item Pr\'eparation \`a l'analyse s\'emantique
\pause
\item Test de grammaticalit\'e
\end{itemize}
\end{frame}

\begin{frame}
\frametitle{\'Etapes d'analyse}
\pause

\textbf{Ex} : \texttt{200*6+137}

\pause
\textbf{Segmentation} : \pause \texttt{[200, *, 6, +, 137]}

\pause
\textbf{Analyse syntaxique}

\Tree [ .E [ .E [ .T [ .T [ .F 200 ] ] * [ .F 6 ] ] ] + [ .T [ .F 137 ] ] ]

\end{frame}

\begin{frame}
\frametitle{Grammaires formelles}

\pause
\textbf{D\'efinition formelle}

$$G = \left< X, V, S, \delta \right>$$

\pause
\begin{itemize}
\item $X$ un ensemble fini de symboles terminaux
\pause
\item $V$ un ensemble fini de symboles non-terminaux
\pause
\item $S \in V$ l'axiome de la grammaire
\pause
\item $\delta$ un ensemble de r\`egles de r\'e\'ecriture
\end{itemize}

\pause
\textbf{Forme normale de Chomsky}
\pause
\begin{itemize}
\item R\`egles lexicales
\begin{itemize}
\item $NC \rightarrow cornemuse$
\end{itemize}
\pause
\item R\`egles non-lexicales
\begin{itemize}
\item $NP \rightarrow DET\;NC$
\item $NP \rightarrow NPP$
\end{itemize}
\end{itemize}
\end{frame}

\begin{frame}
\frametitle{PCFG : grammaires probabilistes}
\pause
\textbf{D\'efinition formelle}

$$G = \left< X, V, S, \delta, \textcolor{red}{P} \right>$$

\pause
\textbf{Probabilit\'e d'une r\`egle}

$$P(X = A \rightarrow \alpha) = P(\alpha \mid A)$$

\pause
\textbf{Probabilit\'e d'un arbre de d\'erivation}

$$P(T = t) = \prod_{i=1}^{n} P(\alpha_i \rightarrow \beta_i)$$

\end{frame}

\begin{frame}
\frametitle{\'Etiquettage morpho-syntaxique (tagging)}
\pause
\textbf{Objectif} :

\begin{center}Assigner \`a chaque symbole terminal un symbole non-terminal\end{center}

\vspace{1cm}

\pause
\textbf{Strat\'egies} :
\begin{itemize}
\pause
\item Utiliser les r\`egles lexicales de la grammaire
\pause
\item Faire appel \`a un tagger externe
\end{itemize}
\end{frame}

\end{document}